\documentclass[a4paper,11pt]{article}
\pagenumbering{arabic}
\usepackage{../environment}
\usepackage{listings}
\usepackage{color}
\usepackage{fancyvrb} % to have verbatim input from solutions.py


\definecolor{dkgreen}{rgb}{0,0.6,0}
\definecolor{gray}{rgb}{0.5,0.5,0.5}
\definecolor{mauve}{rgb}{0.58,0,0.82}

\lstset{frame=tb,
  language=python,
  aboveskip=3mm,
  belowskip=3mm,
  showstringspaces=false,
  columns=flexible,
  basicstyle={\small\ttfamily},
  numbers=none,
  numberstyle=\tiny\color{gray},
  keywordstyle=\color{blue},
  commentstyle=\color{dkgreen},
  stringstyle=\color{mauve},
  breaklines=true,
  breakatwhitespace=true,
  tabsize=3
}

\begin{document}

\begin{center}
    \huge{Solutions to Sheet 5}
\end{center}

\exercise{1}
Let $m \in \Z \setminus\{0, \pm 1\}$ be a squarefree integer. Using Eisenstein's
criterion, one shows that $X^3 - m \in \Q[X]$ is irreducible. Let $K =
\Q[X]/(X^3 -m)$ and write $x$ for the image of $X$ in $K$, so that $x^3 = m$.
Let $z = a + bx + cx^2$. We know that $\NormKQ(z) = a^3 + mb^3 + m^2 c^3 +
3mabc$ from (ii) and $3\cO_K \subset \Z[x]$ from (iv). 
\begin{enumerate}
    \item[(v)] Using (ii) and (iv), show that $\cO_K = \Z[x]$ if $3 \nmid m$
        and $m \not \equiv \pm 1 \mod 9$. 
\end{enumerate}

\textbf{Solution.}

\begin{itemize}[wide, labelindent=0pt]
\item[(v)] Let $z = a + bx + cx^2 \in \cO_K$. We already know that 
    $3z \in \Z[x]$, so the only thing that can go wrong is that there
    are (single) 3's in the denominator's of $a$, $b$ or $c$. 
    We also know that $\NormKQ(3z) = 3^{[K:\Q]}\NormKQ(z) \in 27\Z$. 
    Write $a' = 3a$, $b' = 3b$ and $c' = 3c$. Now $a', b', c' \in \Z$ and we obtain
    \begin{equation*}
    a'^3 + mb'^3 + m^2 c'^3 - 3ma'b'c' \equiv 0 \pmod {27}.
    \end{equation*}
    One can check that there are no non-trivial solutions (that is, no
    solutions without each number divisible by 3).
    Below (and in the github repository) is a snippet of \href{solutions.py}{python3
    code} that does exactly that.
    \lstinputlisting{solutions.py}
    Executing this, we find that there are no solutions. 

    Alternatively, looking at the equivalence mod $3$ gives conditions to $a',
    b', c' \pmod 3$, and one can deduce that there are only trivial solutions
    that way.
\end{itemize}

\exercise{2}
To the right, you do not see the flag of Nepal. The ration of its height to its 
width is equal to a number $\alpha \in \R$ such that $K \coloneqq \Q(\alpha)
= \Q(\sqrt{59-24\sqrt 2})$.
\begin{enumerate}
    \item[(iv)]  Using a computer, one finds that 
        \begin{equation*}
            \Delta_{K/\Q} \left(1, \alpha, \sqrt 2, \sqrt 2 \alpha \right)
            = 2^{10} \cdot 17 \cdot 137
        \end{equation*}
        Deduce from this together with (iii) that 
        $2 \cO_K \subset \cO_F[\beta] = \cO_F + \beta \cO_F$. 
    \item[(v)]  Let $\xi = x + y\beta \in \cO_K$ with $x,y \in F$. 
        We know from $(iv)$ that $x, y \in \frac 12 \cO_F = \allowbreak
        \frac 12 \Z + \frac 1{\sqrt 2} \Z$. Deduce from 
        $\Norm_{K/F}(\xi) \in \cO_F$ that $x^2 - \sqrt 2 xy - y^2 \in \cO_F$. 
    \item [(vi)] Using (iv) and (v), show that $\cO_K = \cO_F[\beta]$. 
\end{enumerate}

\textbf{Solution.}
\begin{enumerate}[labelindent=0pt, wide]
    \item[(iv)]  There are quite a few symbols flying around, let's collect them.
        We have $\alpha = \sqrt{59 - 24\sqrt 2}$, $\beta =
        \frac{\alpha-1}{\sqrt 2}$, in particular $\cO_F[\alpha] \subset
        \cO_F[\beta]$.
        We defined $F = \Q(\sqrt 2)$, and we know from the lecture that
        $\cO_F = \Z[\sqrt 2]$. Part (iii) showed that 
        $2 \alpha^2 \cO_K \subset \cO_F[\beta]$.
        We know that
        $$\Delta_{K/\Q}(1, \alpha, \sqrt 2, \sqrt 2 \alpha) \cO_K \subset \Z +
        \alpha \Z + \sqrt 2 \Z + \sqrt 2 \alpha \Z = \cO_F + \alpha \cO_F
        \subset \cO_F[\beta].$$
        Write $M \coloneqq \Z + \alpha \Z + \sqrt 2 \Z + \sqrt 2 \alpha \Z$.
        Basically by the results about the interplay of linear transformations
        and the discriminant, we find (this is also in the lecture)
        \begin{equation*}
            \Delta_{K/\Q}(1, \alpha, \sqrt 2, \sqrt 2 \alpha) 
            = [\cO_K:M]^2 \Delta_{K/\Q} = 2^{10} \cdot 17 \cdot 137.
        \end{equation*}
        From this we find $[\cO_K:M] \mid 2^5=32$. 
        Let $J \subset \cO_F$ be the ideal defined by $\{r \in \cO_F
        \mid \forall x \in \cO_K: rx \in \cO_F[\beta]\}$. Part (iii) and 
        the above show that $(32, 2\alpha^2) \subset J$. Now we just have to
        calculate:
        \begin{multline*}
            (32, 2\alpha^2) = (32, 2 (59 - 24 \sqrt 2)) = (32, 4(59 - 24 \sqrt 2),
            2\alpha^2) \\ = (32, 4\cdot 59 - 3 \cdot 32\sqrt 2, 2 \alpha^2)  = 
            (4, 2(59 - 24 \sqrt 2)) = (2)
        \end{multline*}
        Hence $2 \subset (J)$ and we are done. 

    \item[(v)] This is just a matter of calculating $\Norm_{K/F}(x + y\beta)$.
        First note that $K = F(\alpha)$ is of degree $2$, with Galois group
        generated by the morphism generated by $\alpha \mapsto -\alpha$. 
        We have
        \begin{equation*}
            \sigma(x + y\beta) = x + y\sigma \left( \tfrac{\alpha-1}{2} \right)
            = x - y\tfrac{1}{\sqrt 2} - y\tfrac{\alpha}{\sqrt 2}
        \end{equation*}
        and one calculates that 
        \begin{multline*}
            \Norm_{K/F}(x+y\beta) = (x - \tfrac{y}{\sqrt 2})^2 - \tfrac{\alpha^2
            y^2}2 = x^2 - \sqrt 2 xy + \tfrac{y^2}2 - \tfrac{(59 - 24\sqrt 2)}2 y^2
            \\
            = \underbrace{x^2 - \sqrt 2 xy - y^2}_{\therefore \in \cO_F} - 
            \underbrace{28y^2 + \tfrac{24}{\sqrt 2}y^2}_{\in \cO_F}.
        \end{multline*}
        Here we used that $y \in \frac 12 (\Z + \sqrt 2 \Z)$, so that 
        $y^2$ has at most a four in its denominator.

    \item[(vi)] The inclusion $\cO_F[\beta] \subset \cO_K$ is already
        known. For the reverse inclusion, take any $\xi = x + y\beta$ as in (v).
        Now we know that $x^2 - \sqrt 2 xy - y^2 \in \cO_F = \Z + \sqrt 2 \Z$. 
        Write $2x = x_1 + \sqrt 2 x_2$ and $2y= y_1 + \sqrt 2 y_2$. The integrality 
        condition yields two equations modulo $4$, as we have
        \begin{multline*}
            \Norm_{K/F}(2\xi) = x_1^2 + 2x_2^2 - 2(x_1 y_2 + x_2 y_1) - y_1^2 -
            2y_2^2 \\ 
            + \sqrt 2\bigg ( 2x_1x_2 - x_1 y_1 - 2x_2y_2 -2y_1y_2 \bigg) \in
            4(\Z +
            \sqrt 2 \Z).
        \end{multline*}
        One can again ask a computer if this has a non-trivial solution, and the
        computer will say \texttt{no}:
        \lstinputlisting{solutions2.py}
        

\end{enumerate}

\exercise{3} Find an integral domain $R$ and a non-zero prime ideal
$\mathfrak P \subset R$ such that $\mathfrak P^{-1} = R$.

\textbf{Solution.}
First remember what the inverse ideal was. If $I \subset R$ is any ideal, 
then $I^{-1} = \{r \in \Frac(R) \mid \forall x \in I: r x \in R  \}$. From here
we can directly see that $\mathfrak P$ cannot be a principal Ideal; if 
$\mathfrak P = (p)$ for some $p \in R$, $\mathfrak P^{-1}$ would simply be given
by $\mathfrak P^{-1}= p^{-1} R \neq R$. Similarly, it was part of the lecture
that for any prime $\mathfrak P$ of a Dedekind domain $R$, $\mathfrak P^{-1}
\supsetneq R$.\footnote{Thanks to Prof. Dill for pointing this out!} So let's
consider the simplest non-principal ideal we know: The ideal $(x,y) \subset
k[x,y]$ for some field $k$. 
It is prime (even maximal, why?\footnote{Because $k[x,y]/(x,y) \cong k$ is a
field.}), and we find that
\begin{equation*}
    (x,y)^{-1} = \{r \in k(x,y) \mid rx \in k[x,y] \wedge ry \in k[x,y]\}
    = x^{-1}k[x,y] \cap y^{-1}k[x,y] = k[x,y].
\end{equation*}
Thus, we have found an example.


\exercise{4} Let $I$ denote the ideal $(2, 1 + \sqrt{-3})$ of the ring
$\Z[\sqrt{-3}]$. 
\begin{enumerate}
    \item Show that $I \neq (2)$. 
    \item Show that $I^2 = 2I$. 
\end{enumerate}
$\Z[\sqrt{-3}]$ is not a Dedekind domain.

\textbf{Solution.}
\begin{enumerate}[labelindent=0pt, wide]
    \item There are many ways to solve this, but let's look at residual rings, 
        that is, calculate $\Z[\sqrt{-3}]/I$. We have $\Z[\sqrt{-3}] \cong
        \Z[X]/(X^2 - 3)$, hence $$\Z[\sqrt{-3}]/I \cong 
        \Z[X]/(X^2 - 3, 2, 1+X) \cong (\Z/2\Z)[X]/(X^2-1, X+1) \cong 
        (\Z/2\Z).$$
        Meanwhile, we have 
        \begin{equation*}
            \Z[\sqrt{-3}]/(2) \cong (\Z/2\Z)[X]/(X^2 - 1) \cong 
            (\Z/2\Z)[X]/(1-X)^2 \cong (\Z/2\Z)^{\oplus 2}.
        \end{equation*}
        Here, the last ismorphism is to be taken as an isomorphism of modules. 
        In particular, we find $I \neq (2)$, as the corresponding residue rings
        are not isomorphic. 
    \item We have $I^2 = (4, 2 + 2 \sqrt{-3}, -2 + 2 \sqrt{-3})$, which
        can be seen simply by multiplying generators. But the last 
        generator is the difference between the first two, so we find 
        \begin{equation*}
            I^2 = (4, 2 + 2 \sqrt{-3}, -2 + 2 \sqrt{-3})
            =( 4, 2 + 2 \sqrt{-3}) = 2I.
        \end{equation*}
        Note that this in particular implies that factorization of an ideal
        into prime factors is not unique: $I$ is prime, and 
        one factorization is given by $I^2 = I\cdot I$. But we have
        $I\cdot I = (2)\cdot I$, and $(2) \neq I$. Note however that 
        uniqueness is not the only thing that fails, the ideal $(2)$
        doesn't even have a decomposition into prime ideals. 
\end{enumerate}


\contactend
\end{document}
