\documentclass[a4paper,11pt]{article}
\pagenumbering{arabic}
\usepackage{../environment}
\begin{document}

\begin{center}
    \huge{Solutions to Sheet 10}
\end{center}

\exercise{1}
Let $p\geq 2$ be a prime number and let $K = \Q(\zeta)$ be the $p$-th
cyclotomic field, where $\zeta = \ec^{2 \pi \ic /p} \in \C$. The minimal
polynomial of $\zeta$ over $\Q$ is $\Phi_n(X) = X^{p-1} + \dots + X + 1$.
Let $l_1, \dots, l_n$ be prime numbers such that $l_i \equiv 1$ mod $p$
for all $i$ and set $L = l_1 \cdots l_n$.
\begin{enumerate}
    \item Show that there xists $x \in \Z$ with $\Phi(xLp)>1$. 
    \item Denote by $l$ a prime number that divides $\Phi_p(xLp)$. 
        Show that $l \not \in \{l_1, \dots, l_n\}$ and $l \neq p$. 
    \item Let $\fl$ be a prime ideal of $\cO_K$ containing $l$. Show that 
        $f(\fl|l\Z) = 1$ and deduce that $l \equiv 1$ mod $p$.
    \item Deduce that there exists infinitely many prime numbers $l$
        such that $l \equiv 1$ mod $p$.
\end{enumerate}
\textbf{Solution.}
\begin{enumerate}[labelindent=0pt, wide]
    \item This is simple analysis. The term $X^{p-1}$ dominates and gets arbitrarily 
        large.
    \item One quickly finds $\Phi_p(xLp) \equiv 1$ mod $l_i$ and mod $p$.
    \item Again, this is an application of Dedekind-Kummer. Again, we can apply 
        Dedekind-Kummer with respect to $\zeta$, as $\cO_K = \Z[\zeta]$, i.e.,
        $[\cO_K:\Z[\zeta]] = 1$. Now $\fl$ corresponds to some factor of the 
        decomposition of $\Phi_n(X)$ mod $l$. As $\Phi_n(xLp) \equiv 0$ mod $l$ (i.e.,
        thre is a root), there is at least one linear term in the decomposition
        of $\Phi_n(X)$. Let this term correspond to some prime ideal $\fl' \mid
        l\cO_K$, which now has residue degree $f(\fl'|l) = 1$ (again, by 3.11). 
        But $\Q(\zeta)/\Q$ is Galois, so the residue degrees of primes over $l$ are
        all the same. Hence $f(\fl|l) = 1$. Proposition 40 now yields that $l \equiv 1$ 
        mod $p$.
    \item Given any finite list $l_1, \dots, l_n$ of primes leaving residue  
        $1$ mod $p$, we can take their product $L$ and find some integer 
        $x > 1$ such that $\Phi_n(xLp) > 1$ by part 1. Now any prime $l$ dividing
        $\Phi_n(xLp)$ is not among the $l_i$ and $\neq p$ by part 2, and 
        part 3 shows that $l \equiv 1$ mod $p$. So no finite list of primes $1$ mod $p$
        can contain all such primes.


\end{enumerate}

\exercise{2}
Let $m<0$ be a squarefree integer and set $K =\Q(\sqrt m)$.
\begin{enumerate}
    \item Show that $\NormKQ(x) > \abs{\Delta_{K/\Q}}/4$ for all $x \in \cO_K\setminus \Z$. 
\end{enumerate}
\textbf{Solution.}
\begin{enumerate}[labelindent=0pt, wide]
    \item Remember the formula for the discriminant of quadratic number fields:
        \begin{equation*}
            \Delta_{\Q(\sqrt {m})/\Q} = \begin{cases}
                4m, &\text{ if }m \equiv 2,3 \pmod 4\\
                m, &\text{ if }m \equiv 1 \pmod 4.
            \end{cases}
        \end{equation*}
        If $m \equiv 1$ mod $4$, this has been basically solved by sheet 6, exercise 2.3:
        There we found that for all $x \in \cO_{\Q(\sqrt m)}$ we have
        $$\NormKQ(x) \geq \abs{\frac{m-1}4} > \abs {\frac m4}=
        \abs{\frac{\Delta_{K/\Q}}4}.$$ 
        The case $m \equiv 2,3$ mod $4$ is handled similarly. We have $\cO_K =
        \Z[\sqrt m]$, and $\NormKQ(a + b\sqrt m) = a^2 + mb^2 \geq m = \abs{\Delta_K}/4$.

\end{enumerate}

\exercise{3}
\begin{enumerate}
    \item Show that $\Cl(\Q(\sqrt{-2023})) = \{1\}$.
    \item Show that $\Cl(\Q(\sqrt{-67})) = \{1\}$.
\end{enumerate}
\textbf{Solution.}
The Idea for both calculations is to follow the proof of lemma 4.4 in the lecture
notes. Let $K = \Q(\sqrt m)$ with some squarefree integer $m<0$ identified as a
subfield of $\C$, and let $I \subset \cO_K$ be any ideal. We can follow the
proof of lemma 4.4 verbatim until just before equation (4.1) to obtain a
\emph{reduced $\Z$-basis} of $(a_1, a_2)$ of $I$. That is, we find elements
$a_1, a_2 \in \cO_K$ with $I = a_1 \Z + a_2 \Z$, such that 
\begin{equation*}
    \abs{\frac {a_2}{a_1}} \geq 1, \quad \Re\left(\frac{a_2}{a_1} \right) \leq 1/2
    \quad \text{ and } \quad \Im\left(\frac{a_2}{a_1} \right) \geq 0.
\end{equation*}
just as in the notes we set $\tau = \frac {a_2}{a_1}$ and find that these conditions
relate to $\abs \tau \geq 1$, $\abs{\Re \tau} \leq 1/2$ and 
$\Im(\tau) \geq 0$. In particular, we find $\Im \tau \geq \sqrt 3 /2$. 
Lemma 1.44 reads $\Delta_K(I) = \Norm(I)^2\Delta_K = \Norm(I)^2bm$, where $b=4$ if 
$m \equiv 2,3$ mod $4$ and $b=1$ otherwise. 
Equation (4.1) also goes through, we find $\Delta_K(I) = -4 \abs{a_1}^4 \Im(\tau)^2$.
Combining these equations, we arrive at 
\begin{equation*}
    \Norm(I) \sqrt{\frac{-bm}3} \geq \abs{a_1}^2 = \NormKQ(a_1).
\end{equation*}
As $a_1 \in I$ we find $I \mid a_1 \cO_K$, so there is some ideal $J$ with
$IJ = a_1 \cO_K$ (i.e., $[J]$ is the inverse of $[I]$ in $\Cl(K)$). Now
\begin{equation*}
    \Norm(I) \Norm(J) = \Norm(IJ) = \Norm(a_1 \cO_K) = \NormKQ(a_1) = \abs{a_1}^2 
    \leq \Norm(I) \sqrt{\frac{-bm}3},
\end{equation*}
implying that 
\begin{equation*}
    \Norm(J) \leq \sqrt{\frac{-bm}3}.
\end{equation*}
The hope is now that this is not too large and leaves us with a number of cases 
that we can handle. So let's see.

\begin{enumerate}[labelindent=0pt, wide]
    \item Note that $2023 = 17^2 \cdot 7$, so that really $K = \Q(\sqrt{-7})$. 
        As $-7$ is $1$ mod $4$, we have $b=1$, and we find 
        $\Norm(J) \leq \sqrt {\frac 73} < 2$. There are no prime ideals with
        norm that low (they cannot lie over a integer prime) so the only possibility
        is $J = \cO_K$. But now $[I] = [J] = \id_{\Cl(K)}$, and 
        $\Cl(K) = \{1\}$.
    \item Again, $-67$ is $1$ mod $4$, but it is already squarefree and
        relatively large, so we'll have to make use of Dedekind kummer. But first of all,
        note that again $b =1$, so we find
        \begin{equation*}
            \Norm(J) \leq \sqrt{\frac{67}3} < \sqrt{23} < 5.
        \end{equation*}
        Now let's inspect the primes above $2$ and $3$. 
        The ring $\cO_K$ is generated as $\Z$-module by 
        $\frac{1+\sqrt{-67}}2$, which has minimal polynomial $T^2 + T + 17$ (I think).
        Mod $2$ we have 
        $T^2 + T + 17 \equiv T^2+ T + 1 $, which is irreducible and mod $3$ we have 
        $T^2 + T + 17 \equiv T^2 + T + 2$, which is irreducible. So we find by
        Dedekind-Kummer that both $2$ and $3$ are inert in $\cO_K$, hence the only
        ideal with norm $\leq 4$ is $J = 2\cO_K$, which is principal. In particular, 
        we find that $I$ has to be principal, hence $\Cl(K) = \{1\}$.
\end{enumerate}


\contactend
\end{document}
