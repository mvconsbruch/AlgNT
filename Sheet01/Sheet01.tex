\documentclass[a4paper,11pt]{article}
\pagenumbering{arabic}
\usepackage{../environment}

\begin{document}

\begin{center}
    \huge{Solutions to Sheet 1}
\end{center}

\exercise{1}
Let $n \in \N$ and $\zeta_n = \ec^{2 \pi \ic/n} \in \C$. Recall that 
$\Z[\zeta_n]$ denotes the smallest subring of the field of complex numbers
that contains $\Z$ and $\zeta_n$. Show that $1/3 \not \in \Z[\zeta_n]$. 

\textbf{Solution.} There are multiple ways to show this.
Denote $A_n = \Z[\zeta_n]$. Note that if $1/3 \in A_n$, we'd have 
$\Z[1/3] \subset A_n$ as well. But there is a fundamental difference between
$A_n$ and $\Z[1/3]$. The latter is a finite (free) $\Z$-module while the 
former is not. As $\Z$ is a PID and submodules of finite\footnote{This assumption can be removed, see for example \url{https://math.stackexchange.com/questions/162945/}.} free modules over a PID are free, we have a contradiction.
This implies other differences between the two rings. For example,
$1/3 \in \Z[1/3]$ is not integral over $\Z$, while every element of 
$\Z[\zeta_n]$ is.

\exercise{2}
Here, $\zeta_3$ is as in Exercise 1. For $f \in \N$ we define
\begin{equation*}
    A_f = \left \{ a + fb \frac{\sqrt{-3} + 1}2 \mid a, b \in \Z \right \}.
\end{equation*}
\begin{enumerate}
    \item Show that $A_f \subset A_1 = \Z[\zeta_3]$ isa subring of 
        $\C$ for all $f \in \N$. 
    \item Let $\abs \cdot$ denote the absolute value on $\C$. Show that 
        $\abs \omega^2 \in \Z$ for all $\omega \in \Z[\zeta_3]$. 
    \item Show that the unit group $\Z[\zeta_3]^\times$ is equal to
        $\{\omega \in \Z[\zeta_3] \mid \abs \omega = 1\}$. 
\end{enumerate}
\textbf{Solution.}
\begin{enumerate}
    \item 
    \item Remember that for the absolute value on $\C$ we have
        \begin{equation*}
            \abs {x + \ic y}^2 = (x + \ic y)(x - \ic y) = x^2 + y^2.
        \end{equation*}
        for $f \in \N$ and $a, b \in \Z$ this gives
        \begin{equation*}
            \abs{a + fb \frac{\sqrt{-3}+1}2}^2 = \left (a + \frac f2\right)^2
            + 3\left( \frac {fb}2 \right)^2 = a^2 + af + (fb)^2 \in \Z. 
        \end{equation*}
    \item All units have invertible absolute value, hence we can conclude
        that if $\omega$ is a unit, it has absolute value $1$. This
        shows one implication. But $\abs \omega^2 = 1$  
        implies that $\omega \overline \omega = 1$, hence $\omega^{-1} = 
        \overline \omega \in \Z[\zeta_3]$, which shows the reverse implication.
\end{enumerate}

\exercise{3}

\exercise{4}
Let $x,y \in \Z$ such that $y^2 -y  = x^3$. Show that $(x,y) = (0,0)$ or 
$(x,y) = (0,1)$. 

\textbf{Solution.} 
As $y$ and $y-1$ share no prime factors, the equation $y^2 - y = y(y-1) = x^3$
implies that both $y$ and $y-1$ are cubes. But this implies $y \in \{0, 1\}$,
and it's easy to see that all solutions are of the given form.

\contactend
\end{document}
