\documentclass[a4paper,11pt]{article}
\pagenumbering{arabic}
\usepackage{../environment}

\begin{document}

\begin{center}
    \huge{Solutions to Sheet 1}
\end{center}

\exercise{1}
Let $n \in \N$ and $\zeta_n = \ec^{2 \pi \ic/n} \in \C$. Recall that 
$\Z[\zeta_n]$ denotes the smallest subring of the field of complex numbers
that contains $\Z$ and $\zeta_n$. Show that $1/3 \not \in \Z[\zeta_n]$. 

\textbf{Solution.} There are multiple ways to show this.
 Note that if $1/3 \in \Z[\zeta_n]$, we'd have 
$\Z[1/3] \subset \Z[\zeta_n]$ as well. But there is a fundamental difference between
$\Z[\zeta_n]$ and $\Z[1/3]$. The latter is a finite free $\Z$-module while the 
former is neither finite nor free. As $\Z$ is a PID and submodules of
finite free modules over a PID are finite and free, we have a contradiction.
This implies other differences between the two rings. For example,
$1/3 \in \Z[1/3]$ is not integral over $\Z$, while every element of 
$\Z[\zeta_n]$ is.

\exercise{2}
Here, $\zeta_3$ is as in Exercise 1. For $f \in \N$ we define
\begin{equation*}
    A_f = \left \{ a + fb \frac{\sqrt{-3} + 1}2 \mid a, b \in \Z \right \}.
\end{equation*}
\begin{enumerate}
    \item Show that $A_f \subset A_1 = \Z[\zeta_3]$ is a subring of 
        $\C$ for all $f \in \N$. 
    \item Let $\abs \cdot$ denote the absolute value on $\C$. Show that 
        $\abs \omega^2 \in \Z$ for all $\omega \in \Z[\zeta_3]$. 
    \item Show that the unit group $\Z[\zeta_3]^\times$ is equal to
        $\{\omega \in \Z[\zeta_3] \mid \abs \omega = 1\}$. 
\end{enumerate}
\textbf{Solution.}
\begin{enumerate}
    \item Note that $\zeta_3 = \frac{\sqrt{-3} - 1}2$ (up to choice), and that 
        $1 + \zeta_3 + \zeta_3^2 = 0$. 
        Also note that $A_f = \{a + fb \zeta_3 \mid a, b \in \Z\}$.
        We have $$(a + fb \zeta_3)(c + fd \zeta_3) = ac + f(ad + cb)\zeta_3 -
        f^2bd(1+\zeta_3) \in A_f,$$
        so $A_f$ is a closed under multiplication. We have $A_f \subset A_{f'}$
        whenever $f' \mid f$, 
        and $A_1 = \Z[\zeta_3]$ is a subring of $\C$. 
    \item Remember that for the absolute value on $\C$ we have
        \begin{equation*}
            \abs {x + \ic y}^2 = (x + \ic y)(x - \ic y) = x^2 + y^2.
        \end{equation*}
        for $f \in \N$ and $a, b \in \Z$ this gives
        \begin{equation*}
            \abs{a + fb \frac{\sqrt{-3}-1}2}^2 = \left (a - \frac {bf}2\right)^2
            + 3\left( \frac {fb}2 \right)^2 = a^2 - abf + (fb)^2 \in \Z. 
        \end{equation*}
    \item All units have invertible absolute value, hence we can conclude
        that if $\omega$ is a unit, it has absolute value $1$. This
        shows one implication. But $\abs \omega^2 = 1$  
        implies that $\omega \overline \omega = 1$, hence $\omega^{-1} = 
        \overline \omega \in \Z[\zeta_3]$, which shows the reverse implication.
\end{enumerate}

\exercise{3}
An integral domain $A$ is called Euclidean if there exists a function 
$n: A \setminus \{0\} \to \Z_{\geq 0}$ such that for all $a \in A$ and
$b \in B \setminus \{0\}$ there exist $q, r \in A$ such that 
$a = bq + r$ and either $r = 0$ or $n(r) < n(b)$.
\begin{enumerate}
    \item Show that Euclidean domains are principal ideal domains.
    \item Show that the ring $\Z[\zeta_3]$ is euclidean.
    \item Show that $\Z[\sqrt 2]$ is euclidean.
\end{enumerate}

\textbf{Solution.}
\begin{enumerate}
    \item Let $R$ be a euclidean ring with norm function $\delta$. Let $\fa
        \subset R$ be an ideal, and let $a \in \fa$ be an element such that 
        $\delta(a)$ is minimal among all elements of $\fa$. Now we have 
        $\fa = (a)$. Indeed, if $f \in \fa$ is another element, we have 
        $f = qa + r$ with $q \in A$ and either $\delta(r) < \delta(a)$ or 
        $r = 0$. As $r = f - qa \in \fa$ and $\delta(a)$ is already minimal
        among elements in $\fa$, $\delta(r) < \delta(a)$ is not possible.
        Therefore we find $r = 0$, hence $f = qa \in (a)$. 

    \item[2.\& 3.] We show that $\nu: z \mapsto \abs{N(z)}$ is a euclidean norm
        function in both cases (where $N$ denotes the respective norm function).
        Write $\cO_K$ for the respective rings. Let $a,
        b \in \cO_K$, $b \neq 0$. We want to show that there 
        are $r \in \cO_K$ and $q \in \cO_K$ with $\nu(r) < \nu(b)$
        and $a = qb + r$.
        The idea is simple. We try to approximate $\frac ab \in K = \Frac(\cO_K)$ by
        some algebraic integer
        $q \in \cO_K$ such that $\abs{N(\frac ab - q)} < 1$. Once we
        found such a $q$, we set $r = a - qb \in \cO_K$ and find
        \begin{equation*}
            \nu(r) = \abs{N(r)} = \abs{N(b)N\left(\frac ab -q \right)}
            < \abs{N(b)} = \nu(b),
        \end{equation*}
        which finishes the proof. 

        So we really only need to show that for $\cO_K = \Z[\zeta_3]$ and 
        $\cO_K = \Z[\sqrt{2}]$, there are such elements $q$. In our cases,
        this is realtively simple. In the case of $\Z[\sqrt 2]$ we write
        $\frac ab = u + v \sqrt 2 $ and choose $x, y \in
        \Z$ such that 
        $\abs{x-u} \leq 1/2$ and $\abs{y-v} \leq 1/2$. Now 
        $$\abs{N(\tfrac ab - q)} \leq \abs{(x-u)^2 - 2(y-v)^2} \leq \tfrac 34 < 1,$$ 
        and we are done. The case $\cO_K = \Z[\zeta_3]$ works the same way.
        Here we find 
        $$\abs{N(\tfrac ab - q)} = \abs{(x-u)^2 + (x-u)(y-v) + (y-v)^2} \leq \tfrac
        34 < 1.$$


        
\end{enumerate}


\exercise{4}
Let $x,y \in \Z$ such that $y^2 -y  = x^3$. Show that $(x,y) = (0,0)$ or 
$(x,y) = (0,1)$. 

\textbf{Solution.} 
As $y$ and $y-1$ share no prime factors, the equation $y^2 - y = y(y-1) = x^3$
implies that both $y$ and $y-1$ are cubes. But this implies $y \in \{0, 1\}$,
and it's easy to see that all solutions are of the given form.

\contactend
\end{document}
        small. 
        
    \item 
        We write $\frac ab = u + v\sqrt d$ and choose $x,y \in \Z$ such that
        $\abs{u-x} \leq 1/2$ and $\abs{v-y} \leq 1/2$. With $q = x + y \sqrt
        d$, we get
        \begin{equation*}
            \abs{N\left(\frac ab - q\right)} = \abs{(u-x)^2 - d(v-y)^2} < 1.
        \end{equation*}
        If we set $r = a - bq \in \Z[\sqrt d]$, we obtain
