\documentclass[a4paper,11pt]{article}
\pagenumbering{arabic}
\usepackage{../environment}

\begin{document}

\begin{center}
    \huge{Solutions to Sheet 3}
\end{center}

\exercise{1}
\begin{enumerate}
    \item Show that $\cO_K^\times = \{x \in \cO_K \mid \Norm_{K/\Q} = \pm 1\}$.
    \item Suppose that $K = \Q(\sqrt m)$ for some negative squarefree integer
        $m$. Determine $\cO_K^\times$. 
\end{enumerate}

\textbf{Solution.}
\begin{enumerate}
    \item We know from the lecture that for any $x \in \cO_K$, the norm
        $\Norm_{K/\Q}(x)$ lies in $\Z$. It is easy to check (for example by 
        defining the norm via the determinant) that the norm induces a
        homomorphism of groups $\Norm_{K/\Q}: \cO_K^\times \to \Z^\times$.
        The claim follows.
    \item 
        Note that $K/\Q$ is always an imaginary extension, so there is an
        embedding $K \inj \C$ (well-defined up to complex conjugation), and
        $\sigma \in \Gal(K/\Q)$ is just given by complex conjugation. Moreover,
        the norm is simply given by the square of
        the complex norm.
        Write $x = a + b \alpha \in \cO_K$, where $a, b \in \Z$ and 
        \begin{equation*}
            \alpha =    \begin{cases}
                            \sqrt m \quad &\text{ if } m \equiv 2,3 \pmod 4, \\
                            \frac{1+\sqrt m}2 \quad &\text{ if } m \equiv 1 \pmod 4.
                        \end{cases}
        \end{equation*}
        In the first case, the norm computes as
        \begin{equation*}
             \NormKQ (a + b \alpha) = (a + b \alpha)(a + b
            \sigma(\alpha)) =  a^2 - mb^2,
        \end{equation*}
        where $\sigma \in \Gal({K/\Q})$ is the non-trivial element (acting by 
        complex conjugation after choosing a complex embedding).
        In the second case we find similarly
        \begin{equation*}
             \NormKQ (a + b \alpha) = (a + b \alpha)(a + b
            \sigma(\alpha)) =  a^2 + ab + b^2 \frac{(1-m)}4.
        \end{equation*}
        In both cases the norm is greater than $0$, and we could try to solve
        the exercise by solving the equations
        $\NormKQ(a + b\alpha) = 1$ eplicitely. But using the triangle
        inequality, we can save a lot of work. We find that every
        unit $x \in \cO_K^\times$ must have trace $\abs{\Tr_{K/\Q}(x)} = \abs{x +
        \sigma(x)} \leq 2$. Remember that trace and norm also arise as coefficients
        of the characteristic polynomial of $x$, and hence every unit
        $x \in \cO_K^\times$ satisfies
        \begin{equation*}
            x^2 - \Tr_{K/\Q}(x) x + \NormKQ(x) = x^2 - \Tr_{K/\Q}(x) x + 1 = 0.
        \end{equation*}
        As the trace of $x$ over $\Q$ is always an integer, we find 
        $\Tr_{K/\Q}(x) \in \{-2,-1,0,1,2\}$. 
        Now there are three tracases:
        \begin{itemize}
            \item $\Tr(x) = \pm 2$. In this case $x^2 \mp 2x + 1 = (x \mp 1)^2$
                and $x = \pm 1$. 
            \item $\Tr(x) = 0$. In this case $x$ satisfies $x^2 = -1$, hence
                $x = \pm \ic$. It is easy to check that $\ic \in \cO_K$ iff 
                $m = -1$. 
            \item $\Tr(x) = \pm 1$. In this case $x$ is a third of a sixth root of
                unity. Indeed, if $\Tr(x) = -1$ we find $0 = (x-1)(x^2 + x + 1)
                = x^3 - 1$, so $x$ is a third root of unity. If $\Tr(x) = 1$ we
                find $0 = (x+1)(x^2 - x + 1) = x^3 + 1$, so $x$ is a sixth root
                of unity. Note that we have already seen that $\zeta_3 \in
                \cO_{\Q(\sqrt{-3})}$, and $\zeta_6 = \frac 12 +
                \frac{\sqrt{-3}}2$ also lies in this ring of integers.  
        \end{itemize}
        Finally, it is not hard to see that two non-isomorphic quadratic number
        fields have trivial intersection (after choosing embeddings into $\C$).
        This shows that we have fully characterized the units of
        the ring of integers of $\Q(\sqrt m)$ for negative square-free $m$. 



\end{enumerate}

\exercise{2}
Let $K$ and $L$ be number fields and let $\phi: K \to L$ be a ring homomorphism.
Show that $\phi(\cO_K) \subset \cO_L$.

\textbf{Solution.} We know that $\cO_L$ is the integral closure of $\Z$ in $L$. This
means $\cO_L$ is the subring of elements in $L$ that arise as roots of polynomials
in $\Z$. The same is true for $\cO_K$ in $K$. If any $x \in \cO_K$ is a root of a
monic polynomial $f_x(T) \in \Z[T]$. Then $\phi(x) \in L$ is a root of $f$ as well,
as $f(\phi(x)) = \phi(f(x)) = 0$ (remember that any ring morphism is a
homomorphism of abelian groups. In particular, $\phi$ is the identity
on $\Z$, and thereby does not change the coefficients of $f$).

\exercise{3}
Let $m \in \Z \setminus \{0, \pm 1\}$ be a squarefree integer. Using Eisenstein's
criterion, one shows that $X^3 -m \in \Q[X]$ is irreducible (you do not need to check
this). Set $K = \Q[X]/(X^3-m \Q[X])$, we write $x$ for the image of $X$ in $K$
so that $x^3 = m$. 
\begin{enumerate}
    \item Show that $\Delta_{K/\Q}(1,x,x^2) = -3^3 m^2$.
    \item Let $a,b,c \in \Q$. Compute $\Norm_{K/\Q}(a + bx + cx^2)$.
\end{enumerate}

\textbf{Solution.} 
\begin{enumerate}
    \item The Galois group of $K$ over $\Q$ is of degree $3$ and generated by the 
        morphism sending $x$ (a primitive element of $K$) to $\zeta_3 x$, at least 
        after embedding $K$ into $\C$ (say). By Lemma 1.32 in the script we obtain
        \begin{equation*}
            \Delta_{K/\Q}(1,x,x^2) = \det \begin{pmatrix}
                                            1 & x & x^2 \\
                                            1 & \zeta_3 x & \zeta_3^2 x \\
                                            1 & \zeta_3^2 x & \zeta_3 x^2
                                          \end{pmatrix}^2.
        \end{equation*}
        The determinant of the matrix is readily computed to $3x^3(\zeta_3^2 - 
        \zeta_3)$, which has square $9x^6(-3) = -3^3m^2$, as desired.

    \item Let $\alpha = a + bx + cx^2$. Let $B$ be the basis $(1,x,x^2)$ of $K$
        as a $\Q$ vector space. Then $\alpha$ sends $1$ to the vectors
        $(a,b,c)$, $x$ to the vector $(mc, a, b)$ and $x^2$ to the vector
        $(mb, mc, a)$. We find that as a matrix with respect to $B$, 
        multiplication by $\alpha$ is given by
        \begin{equation*}
            \begin{pmatrix} 
                a  & mc & mb \\
                b  & a  & mc  \\
                c  & b  & a
            \end{pmatrix},
        \end{equation*}
        and the determinant of this matrix is (hopefully)
        \begin{equation*}
            a^3 + mb^3 + m^2 c^3 - 3mabc.
        \end{equation*}
        This is $\NormKQ(\alpha)$.
\end{enumerate}

\exercise{4}
To the right, you do not see the flag of Nepal. The ration of its height to its 
width is equal to a number $\alpha \in \R$ such that $K \coloneqq \Q(\alpha)
= \Q(\sqrt{59-24\sqrt 2})$.
\begin{enumerate}
    \item Show that $[K:\Q]=4$ and that 
        \begin{equation*}
            \left(1, \sqrt{59 - 24\sqrt 2}, \sqrt 2, \sqrt2 \sqrt{59- 24\sqrt 2}
                \right)
        \end{equation*}
        is a $\Q$-basis of $K$.
    \item Show that $\beta \coloneqq (-1 + \sqrt{59 - 24\sqrt 2}/\sqrt 2 \in \cO_K$. 
    \item Set $F = \Q(\sqrt 2)$. Show that $2(59- 24\sqrt 2) \cO_K \subset
        \cO_F[\beta]$. 
\end{enumerate}


\contactend

\end{document}
