\documentclass[a4paper,11pt]{article}
\pagenumbering{arabic}
\usepackage{../environment}
\begin{document}

\begin{center}
    \huge{Solutions to Sheet 9}
\end{center}

\exercise{1}
Let $K = \Q(\zeta_8)$.
\begin{enumerate}
    \item Show that $K = \Q(\sqrt 2, \ic)$. 
    \item Let $p$ be an odd prime number. Show that $\legendre 2p = 1$ if and only if
        $p \equiv 1,7 \pmod 8$.
\end{enumerate}
\textbf{Solution.}
\begin{enumerate}[labelindent=0pt, wide] 
    \item Note that $\zeta_8 = \frac 1{\sqrt 2} + \frac \ic {\sqrt 2}$, so that we have
        $\Q(\zeta_8) \subset K$. But we have $\deg \Q(\zeta_8) = \phi(8) = 4$ and 
        also $\deg \Q(\sqrt 2, \ic) 4$, which implies equality. There are other many
        ways to do this.
    \item We have seen that $\legendre 2p = 1$ if and only if $p$ splits totally in
        $\Q(\sqrt 2)$, i.e., $p = \fp_1 \fp_2$ for two distinct prime ideals of 
        $\cO_{\Q(\sqrt 2)}$. Since the discriminant of $\Q(\sqrt 2) = 8$, we also 
        know that every odd prime is unramified in $\Q(\sqrt 2)$ (as a prime ramifies
        iff it divides the discriminant).

        But being split totally is equivalent to having 
        frobenius element equal to the identity: For every prime 
        $\fp \subset \cO_K$ over $p \Z$ we have that $\#D(\fp|p) = e(\fp|p)f(\fp|p) = 
        f(\fp|p)$, and we obtain 
        \begin{equation*}
            \text{$p>2$ totally split in $\Q(\sqrt 2)$} \iff \forall \fp \mid
            p\cO_{\Q(\sqrt 2)}: f(\fp|p) = 1 \iff \forall \fp \mid
            p\cO_{\Q(\sqrt 2)}:\#D = 1.
        \end{equation*}
        Since $p$ is unramified we know that for every $\fp \mid p\cO_K$,
        $D(\fp|p) \cong \Gal(\cO_{\Q(\sqrt 2)}/\fp | \Z/p\Z)$, and the latter
        is generated by the Frobenius element. In particular, we find that (by
        definition), $D(\fp|p)$ is generated by the generalized Frobenius
        element $\left(\frac{\Q(\sqrt 2)/Q} \fp \right)$, which by 
        Definition-Lemma 3.31 is isomorphic to the restriction of
        $\left( \frac{K/\Q}{\fp}\right)$ to $\Q(\sqrt 2)$. 
        We are now almost done, because we know how to compute the Frobenius element 
        in $K$! It is given by $\zeta_8 \mapsto \zeta_8^p$. Now we use that 
        $\sqrt 2 = \zeta_8 + \zeta_8^7$, and one readily checks that 
        \begin{equation*}
            \left( \frac{K/\Q}{\fp}\right)(\sqrt 2) = \left(
            \frac{K/\Q}{\fp}\right)(\zeta_8 + \zeta_8^7) = \zeta_8^p +
            \zeta_8^{-p} = \sqrt 2
        \end{equation*}
        if and only if $p \equiv 1,7 \pmod 8$.
        
\end{enumerate}


\exercise{2}
Let $p \geq 2$ be a prime number. Set $K = \Q(\zeta_p)$. Show that 
$\Delta_K = (-1)^{(p-1)(p-2)/2}p^{p-2}$.

\textbf{Solution.}
Note that $\zeta_p, \dots, \zeta_p^{p-1}$ is a $\Z$-basis for $\cO_K$ by results of the
script (Lemma 3.36). Hence it suffices to show that $\Delta_{K/\Q}(\zeta_p , \dots,
\zeta_p^{p-1}) = (-1)^{(p-1)(p-2)/2} p^{p-2}$. We know that 
$\Gal(K/\Q) \cong \{\sigma_i | 1 \leq i < p-1\},$ where $\sigma_i$ is the morphism 
sending $\zeta_p$ to $\zeta_p^i$. 
We have seen that 
$$\Delta_{K/\Q}(\zeta_p, \dots, \zeta_p^{p-1}) = \det A^2,$$ where $A$ is the
matrix with $ij$-th entry given by $\sigma_i(\zeta_p^j)$. Now $A$ is a
Vandermonde matrix, and we obtain
\begin{equation*}
    \det A^2 = \left(\prod_{1 \leq i < j \leq p-1} (\zeta_p^i - \zeta_p^j) \right)^2
    = (-1)^{(p-1)(p-2)/2} \prod_{i \neq j} (\zeta_p^i - \zeta_p^j) = \pm \prod_{i=1}^{p-1}
        \phi_p'(\zeta_p^i) 
\end{equation*}
where $\phi_p(X) = \frac{X^p-1}{X-1}$ is the $p$-th cyclotomic polynomial. 
We can differentiate the equation
\begin{equation*}
    \phi_p(X) (X-1) = X^p-1
\end{equation*}
to find that 
$$\phi_p'(X)(X-1) + \phi_p(X) = pX^{p-1}.$$ 
This gives
\begin{equation*}
    ... = (-1)^{(p-1)(p-2)/2} p^{p-1} \prod_{i=1}^{p-1} \frac 1{1 - \zeta_p^i}
    = (-1)^{(p-1)(p-2)/2} p^{p-1}\phi_p(1)^{-1} =  (-1)^{(p-1)(p-2)/2} p^{p-2}.
\end{equation*}
In the last step we used that $X^p-1 = \prod_{i=0}^{p-1} (X- \zeta_p^i)$, 
so that $\phi_p(X) = \frac{X^p-1}{X-1} = \prod_{i=1}^{p-1} (X- \zeta_p^i)$.




\contactend
\end{document}
