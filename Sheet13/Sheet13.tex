\documentclass[a4paper,11pt]{article}
\pagenumbering{arabic}
\usepackage{../environment}
\begin{document}

\begin{center}
    \huge{Solutions to Sheet 12}
\end{center}

\exercise{1}
Let $I$ be an ideal of a number field $K$. Show that htere is a finite field extension $L$ of $K$ such that 
$I \cO_L$ is a principal ideal of $\cO_L$.

\textbf{Solution.}
By finiteness of $\Cl(K)$ there is some integer $m$ such that $[I]^m = [I^m] = [(1)] \in \Cl(K)$,
i.e., $I^m = (\alpha)$ is a principal ideal. We put $L = K(\alpha^{1/m})$. Now 
$\alpha^{1/m} \in \cO_L$, and we have 
$$(I \cO_L)^m = I^m \cO_L = \alpha \cO_L = (\alpha^{1/m})^m \cO_L.$$
After decomposing $I\cO_L$ and $\alpha^{1/m}\cO_L$ into prime factors, we see that 
this equation implies $I\cO_L = \alpha^{1/m} \cO_L$. 

\exercise{2} 
Let $K = \Q(\sqrt 2, \sqrt 3)$ and set 
\begin{equation*}
    \Gamma = \{(1+ \sqrt 2)^i(2 + \sqrt 3)^j(\sqrt 2 + \sqrt 3)^k \mid i,j,k \in \Z\}.
\end{equation*}
Show that $\Gamma$ is a subgroup of $\cO_K^\times$ and that $[\cO_K^\times: \Gamma] < \infty$.

\textbf{Solution.}
Write $u,v,w$ for the respective factors, so that $\Gamma = u^\Z v^\Z w^\Z$. 
Note that $\NormKQ(u) = \NormKQ(v) = 1$ and $\NormKQ(w) = -1$, so that indeed,
$u,v,w$ are units and $\Gamma$ is a subgroup of $\cO_K^\times$. One quickly verifies that
$K$ is totally real. Indeed, it is Galois and there is a embedding $K \inj \R$ (now all
other embeddings are obtained by shifting with elements in the Galois group). 
Hence, by Dirichlet's unit theorem, 
$$\cO_K^\times \cong \mu(K) \times \Z^{r+s-1} = \pm 1 \times \Z^3.$$
On the other hand, $\Gamma$ is free of rank $3$. Indeed, $u \in \Q(\sqrt 2)^\times$,
$v \in \Q(\sqrt 3)^\times$ and $w \in \Q(\sqrt 2, \sqrt
3)^\times\setminus(\Q(\sqrt 2)^\times \cup \Q(\sqrt 3)^\times) \cup \{1\}$. These multiplicative
subsets of $\Q(\sqrt 2, \sqrt 3)$ only have trivial intersection.

Let $\cO_{K, >0}^\times \cong \Z^3$ be the (free) group of positive units (here we implicitly
fix a inclusion $K \inj \R$). 
Now $\Gamma$ is a free subgroup of full rank this group, and in particular, 
its has finite index. The inclusion $\cO_{K, >0}^\times \inj \cO_K^\times$ also has
finite index, hence $\Gamma \inj \cO_K^\times$ has finite index. 

\exercise{3}
Let $K$ be a totally real number field, i.e., one that has \textbf{only} real embeddings.
Let 
\begin{equation*}
    T \subset \Hom(K, \R) = \{\tau: K \to \R \mid \tau \text { is a field homomorphism}\}
\end{equation*}
be a proper non-mepty subset. Show that there exists $u \in \cO_K^\times$ such that 
$0 < \tau(u) < 1$ for $\tau \in T$ and $\tau(u) > 1$ for $\tau \in \Hom(K, \R)
\setminus T$.

\textbf{Solution.} 
Let $\sigma_1, \dots, \sigma_r: K \to \R$
be the real embeddings of $K$ (in our case $r = n = [K : \Q]$). 
From the proof of Dirichlet's unit theorem, we know that the map
\begin{equation*}
    \cL: \cO_K^\times \to \R^{r}, \quad u \mapsto (\log \abs{\sigma_1(u)}, \dots, \log \abs{\sigma_r(u)})
\end{equation*}
is a group homomorphism from $\cO_K$ to $\R^{r-1}$. It's image lies in the sub vector
space $V$ given by 
\begin{equation*}
    V = \left\{(x_1, \dots, x_r)^t \in \R^r \mid \sum_{i = 1}^r x_i = 0\right\},
\end{equation*}
and its kernel is given by $\mu(K)$, the group of roots of unity in $K$ (in our case this
is $\cong \{\pm 1\}$). Also, the image $\cL(\cO_K^\times)$ has full rank in 
$V$, i.e., $\cL(\cO_K^\times) \otimes_\Z \R \cong V$ (it is a lattice in $V$). 

Without loss of generality we can assume that 
$T = \{\sigma_1, \dots, \sigma_q\}$ for some $1 \leq q < r$. Let $Q \subset
\R^r$ be the quadrant given by
\begin{equation*}
    Q = \{(x_1, \dots, x_r)^t \in \R^r \mid x_i < 0 \text{ for }i = 1, \dots, q
    \text{ and }x_i > 0 \text{ for } i = q+1, \dots, r \}.
\end{equation*}
The intersection $Q \cap V$ is non-empty by construction, and one readily verifies that
there is a point $x \in Q \cap \cL(\cO_K^\times)$. Now choose some preimage
$u \in \cO_K^\times$ of $x$. As $u$ satisfies $\abs{\sigma_i(u)} < 1$ for 
$1 \leq i \leq q$ and $\abs{\sigma_i(u)} > 1$ for $q < i \leq r$,
the element $u^2$ satisfies all constraints.

\exercise{4}
Let $K$ be a number field, let $I$ be a non-zero ideal of $\cO_K$ and let $C \in \Cl(K)$. Use
theorem 5.3 to show that there exists a non-zero ideal $J$ of $\cO_K$ such that $I + J = \cO_K$ and
$C = [J]$. 

\textbf{Solution.}
%Theorem 5.3 counts the number of objects in an ideal class up to some given norm $t$.
%For $C \in \Cl(K)$, let $i(K, C, t)$ be the set of ideal in the given ideal class $C$
%with norm $\leq t$ (just as in the statement). Then theorem 5.3 reads
%\begin{equation*}
%    i(K, C, t) = \kappa t + O(t^{1 - 1/d}) = \kappa t + o(t).
%\end{equation*}
%Here we used \emph{big-$O$-notation}, what this means essentially is that $i(K,C,t)$ is
%of size $\kappa t$ up to some small error. 
%Let's solve the exercise. We will show that the set of ideals
%\begin{equation*}
%    \{J \subset \cO_K \mid J \in C \& J + I = \cO_K \& \Norm(J) \leq t\}
%\end{equation*}
%is non-epty for $t$ sufficiently large. Note that two ideals are coprime if and
%only if they don't share a prime factor. 
%If we assume that $J = \fp$ is prime, this is immediate. We have
\red{will add this later!}

\contactend
\end{document}
