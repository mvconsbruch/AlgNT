\documentclass[a4paper,11pt]{article}
\pagenumbering{arabic}
\usepackage{../environment}
\begin{document}

\begin{center}
    \huge{Solutions to Sheet 13}
\end{center}

\exercise{1}
Let $K$ be a number field of degree $d$.
\begin{enumerate}
    \item Show that there exists a constant $C$, depending only on $K$,
        with the following property. If $I$ is a principal ideal of $\cO_K$,
        then $I = \alpha \cO_K$ for some $\alpha \in \cO_K$ such that 
        $\abs{\sigma(\alpha)} \leq C \Norm(I)^{1/d}$ for every real or 
        complex embedding of $K$.
    \item Let $K \subset \R$ be a quadratic number field. Let $\eta \in \cO_K^\times$
        be such that $\cO_K^\times = \{\pm \eta^n ; n \in \Z\}$. Show that, in this
        case, one can take $C = \max\{\abs \eta, \abs \eta^{-1}\}^{1/2}$.
    \item Do there exists $x,y \in \Z$ with $x^2 - 82y^2 = 2$?
\end{enumerate}
\textbf{Solution.}
\begin{enumerate}[wide, labelindent=0pt]
    \item Assume that $I = (\alpha)$.
        Then $I = (u \alpha)$ for every unit $u \in \cO_K^\times$. Look at the morphism
        \begin{equation*}
            \cL: \cO_K\setminus \{0\} \to \R^{r+s}, 
            \xi \mapsto (\log \abs{\sigma_1(\xi)}, \dots, \log \abs {\sigma_r(\xi)}, 
            2\log \abs{\sigma_{r+1}(\xi)}, \dots, \log \abs{\sigma_{r+s}(\xi)}).
        \end{equation*}
        We have seen in the lecture that the image of $\cO_K^\times$ under this map 
        is a lattice in a linear subspace $V$. Here $V$ is given by those vectors
        of $\R^{r+s}$ whose coordinates sum up to $0$. We are interested
        in the image of the set $\alpha \cO_K^\times$ under $\cL$. This is contained
        in the affine-linear subspace $\cL(\alpha) + V$. 
        
        The exercise now translates to: There is some constant $C > 0$ depending only
        on $K$ such that there is a point $x = (x_1, \dots, x_r, 2x_{r+1}, \dots,
        2x_{r+s}) \in \cL(\alpha) + \cL(\cO_K^\times) \subset \cL(\alpha)+V$ with 
        \begin{equation*}
             \max x_i \leq \frac 1d \log \Norm(I) + \log(C)
        \end{equation*}

        We write $W$ for the vector space spanned by 
        \begin{equation*}
            w_0 = \frac 1d (1,\dots,1).
        \end{equation*}
        This is orthogonal to the subspace $V$. Write $\cL(\alpha) = w + v$ where
        $v \in V$ and $w \in W$. We have 
        \begin{equation*}
            \norm{\cL(\alpha)}_1 = \log \Norm(I),
        \end{equation*}
        hence we obtain 
        $$\norm w_\infty = \frac 1d \norm w_1 \leq \frac 1d \log \Norm(I).$$
        All we need to do is to find a point of $\cL(\alpha \cO_K^\times)$
        close to $w$. For $\nu \in V$ define $d(\nu) = \inf_{\gamma \in \cL(\cO_K^\times)}
        (\norm{\nu-\gamma}_\infty)$, and set $C = \sup_{\nu \in V} d(\nu)$. This 
        is well-defined because $\cL(\cO_K^\times)$ is a lattice in $V$.
        Now $C$ only depends on $K$, and we find a point 
        $$(x_1,\dots, x_r, 2x_{r+1},\dots, 2x_{r+s}) = x = w + \nu_0$$
        with $\nu_0 \in V$ and $\norm{\nu_0}_\infty \leq C$. In particular,
        \begin{equation*}
        \max x_i \leq \norm x_\infty \leq \norm w_\infty + \norm {\nu_0}_\infty \leq
        \frac 1d \log \Norm(I) + C.
        \end{equation*}
        This solves the exercise.
        
    \item Let $\eta$ be as in the question and assume $\abs \eta < 1$. Let
        $\sigma_1, \sigma_2: K \inj \R$ be the two real embeddings of $K$ (i.e., 
        $\sigma_1$ is the identity on $K \subset \R$ and $\sigma_2$ is "conjugation").
        Suppose $I = (\alpha)$. Then we can pick $n$ such that 
        \begin{equation*}
            \sigma_1(\eta^n \alpha) = \abs \eta^n \abs{\sigma_1(\alpha)}
            \in \left(\abs \eta^{1/2} \Norm(I)^{1/2}, \abs \eta^{-1/2}
            \Norm(I)^{1/2}\right).
        \end{equation*}
        Now 
        \begin{equation*}
            \abs{\sigma_2(\eta^n \alpha)} = \abs{\eta^{-n}} \abs{\sigma_2(\alpha)}
            = \left( \abs \eta^{-n} \sigma_1(\alpha)^{-1} \right) \Norm(I)
                \leq \frac 1{\abs{\eta}^{1/2}} \Norm(I)^{1/2}.
        \end{equation*}

    \item Let $K = \Q(\sqrt {82})$. A unit as in 2 is given by $\eta = 9 + \sqrt 82$. 
        This can be checked similarly to sheet 11, exercise 4. 
        Also, note that (by Dedekind-Kummer) $2 \cO_K= \fp^2$ for some prime
        ideal $\fp$. Now $\Norm(\fp)=2$, and a solution to $x^2 - 82y^2 = 2$
        would result in $\fp$ being principal. 
        By part 1 and two, it suffices to show that there is no solution with
        $$\max \abs{x \pm \sqrt {82} y} \leq \Norm(2 \cO_K)^{1/2} \sqrt{9 + \sqrt{82}}
        < 7.$$
        But there are no such solutios as $\sqrt 82 > 7$. 
\end{enumerate}


\exercise{2}
Show that 
\begin{equation*}
    \zeta_{\Q(\sqrt 6} \zeta_{\Q(\sqrt 7} \zeta_{\Q(\sqrt {42})} 
    = \zeta_{\Q(\sqrt 6, \sqrt 7)} \zeta_\Q^2.
\end{equation*}

\textbf{Solution.} 
Here we will only sketch a solution. The details are tedious.
Recall that from the lecture (Example 3.12, say) we have for quadratic fields
$K = \Q(\sqrt m)$ with discriminand $\Delta_K$
\begin{equation*}
    p\cO_K = 
    \begin{cases}
        \textit{prime ideal}, &\text{ if } \legendre{\Delta_K}{p} = -1 \\
        \fp_1 \fp_2, &\text{ (totally split) if } \legendre{\Delta_K}{p} = 1 \\
        \fp^2, &\text{ (totally ramified) if } \legendre{\Delta_K}{p} = 0.
    \end{cases}
\end{equation*}

Also, note that for $K$ as above we have
\begin{multline*}
    \zeta_K(s) = \prod_{\fp} \left( 1 - \frac{1}{\Norm(\fp^s)} \right)^{-1}  = 
    \prod_{p \in \Z \text{ prime}} \prod_{\fp \mid p \cO_K} \left(1-
    \frac1{\Norm(\fp)^s}\right)^{-1} \\
    = \prod_{\legendre{\Delta_K}{p} = -1}\left( 1 - \frac{1}{\Norm(\fp)^{2s}}\right)^{-1}
        \prod_{\legendre{\Delta_K}{p} = 1}\left( 1 - \frac{1}{\Norm(\fp)^{s}}\right)^{-2}
        \prod_{\legendre{\Delta_K}{p} = 0}\left( 1 - \frac{1}{\Norm(\fp)^{s}}\right)
\end{multline*}
This yields some expansion of $\zeta_{\Q(\sqrt 6} \zeta_{\Q(\sqrt 7}
\zeta_{\Q(\sqrt {42})}$ in terms of factors indexed by prime numbers.

Write $L = \Q(\sqrt 6, \sqrt 7)$. We want to do something similar as above for the
function $\zeta_{\Q(\sqrt 6, \sqrt 7)}$. Recall that if $p \cO_K = \fp_1^{e_1} \cdots
\fp_g^{e_g}$ we have $e_1 = \dots = e_g = e$, $f(\fp_1) = \dots = f(\fp_g) = f$ and
$efg = 4$. Also, as $\Gal(L/\Q) \cong (\Z/2\Z)^2$ is not cyclic, we cannot have 
inert primes, i.e., we never have $f=4$. Indeed, if there was an inert prime $\fp \mid p$,
we'd find $\Gal(L/\Q) = D(\fp|p) \cong \Gal(\kappa(\fp)/\FF_p)$, and the latter 
(being the Galois group of an extension of finite fields) is cyclic. 
Hence, the only possible splitting behaviours of a prime $p \in \Z$ are:
\begin{equation*}
    \begin{aligned}
        g = 4 &\implies \prod_{\fp \mid p}\left(1 - \frac1{\Norm(\fp)^s}\right)^{-1} = (1 - p^{-s})^{-4} \\
        f = 2, g = 2 &\implies \prod_{\fp \mid p}\left(1 - \frac1{\Norm(\fp)^s}\right)^{-1} = (1 - p^{-2s})^{-2} \\
        e = 2, f = 2, g = 1 &\implies \prod_{\fp \mid p}\left(1 - \frac1{\Norm(\fp)^s}\right)^{-1} = (1 - p^{-2s})^{-1} \\
        e = 2, f = 1, g = 2 &\implies \prod_{\fp \mid p}\left(1 - \frac1{\Norm(\fp)^s}\right)^{-1} = (1 - p^{-s})^{-2} \\
        e = 4, f = 1, g = 1 &\implies \prod_{\fp \mid p}\left(1 - \frac1{\Norm(\fp)^s}\right)^{-1} = (1 - p^{-s})^{-1}.
    \end{aligned}
\end{equation*}
But one easily checks that $\Delta_L$ has only prime divisors $2,3,7$, so that all other
primes are unramified on $\cO_L$. Using this and that
\begin{equation*}
    \legendre{42}p= \legendre6p \legendre 7p,
\end{equation*}
we can show that the Euler factors at each prime $p$ coincide for both functions.

\exercise{3}
Let $K$ be a number field and let 
\begin{equation*}
    \log: \C \setminus (-\infty, 0] \to \C
\end{equation*}
denote the principal branch of the complex logarithm. The variable $\fp$ always
runs over the non-zero primes of $\cO_K$ in the following.
\begin{enumerate}
    \item Show that 
        \begin{equation*}
            \lim_{s \rightarrow 1^+} \frac{1}{\log(s-1)} \sum_\fp \log (1 -
            \Norm(\fp)^{-s}) = 1.
        \end{equation*}
    \item Show that 
        \begin{equation*}
            \sum_{\fp;\ f(\fp)>1} \frac 1{\Norm(\fp)} + \sum_\fp \sum_{n=2}^\infty \frac 1{n \Norm(\fp)^n} < \infty.
        \end{equation*}
    \item Using 1 and 2, deduce that there exists infinitely many prime ideals $\fp$ of $\cO_K$ with 
        $f(\fp)=1$. 
\end{enumerate}
\textbf{Solution.}
The following is a bit unprecise as I sometimes forgot to insert absolute-value
brackets. But it works out if we simply assume $s \in \R$ everywhere.
\begin{enumerate}
    \item We know that the Dedekind-zeta function $\zeta_K(s)$ is holomorphic
        in $\Re s >1$, has a pole of order $1$ at $s=1$ with residue $\kappa >0$
        and has an Euler product
        \begin{equation*}
            \zeta_K(s) = \prod_{\fp} \left(1- \Norm(\fp)^{-s}\right)^{-1}.
        \end{equation*}
        Hence we find
        \begin{equation*}
            \lim_{s \to 1^+} \left((s-1) \prod_\fp \left(1- \Norm(\fp)^{-s}\right)^{-1} 
                \right)
            = \kappa,
        \end{equation*}
        and the claim follows after taking logarithms.
    \item Note that there are at most $[K:\Q]$ prime ideals of $\cO_K$ above
        each prime number $p \in \Z$. If $\fp$ lies above $p$ and $f(\fp)>1$,
        we have (by definition) $\Norm(\fp)\geq p^2$. Hence we obtain that 
        \begin{equation*}
            \sum_{\fp, f(\fp)>1} \Norm(\fp)^{-s} \leq 
            [K:\Q] \sum_{p} p^{-2s}, 
        \end{equation*}
        which is (absolutely) convergent for $\Re s > \frac 12$. Similarly, we find that 
        \begin{equation*}
            \frac{1}{[K:\Q]} \sum_\fp \sum_{n=2}^\infty \frac 1{n \Norm(\fp)^{sn}}
            \leq \sum_p \sum_{n=2}^\infty \frac 1{np^{sn}}
            < \sum_p p^{-2s} \sum_{n=0}^\infty p^{-sn} 
            < \left( \sum_{n=0}^\infty 2^{-sn}\right) \sum_p p^{-2s} .
        \end{equation*}
        This is absolutely convergent for $\Re s > 0$.

    \item Recall that $\log(1+t) = \sum_{n=1}^\infty \frac {(-1)^{n-1}}n t^{n}$ for
        $\abs t < 1$. We plug this
        into the logarithm of the euler product to obtain for $\Re s > 1$
        \begin{equation*}
            \log \zeta_K(s) = -\sum_\fp \log(1- \Norm(\fp)^{-s})
            = \sum_\fp \sum_{n=1}^\infty \frac 1n \Norm(\fp)^{-ns}.
        \end{equation*}
        Splitting off the $n=1$ terms, this yields
        \begin{equation*}
            \log \zeta_K(s) = \sum_{\fp, f(\fp) = 1} \Norm(\fp)^{-s} 
            + \sum_{\fp, f(\fp)>1} \Norm(\fp)^{-s} + 
            \sum_\fp \sum_{n=2}^\infty \frac{1}{n \Norm(\fp)^{sn}}.
        \end{equation*}
        The left hand side of this equation diverges for $s \to 1^+$, but the last 
        two terms of the RHS remain finite by part 2. The claim follows.
        
\end{enumerate}

\contactend
\end{document}
