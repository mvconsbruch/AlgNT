\documentclass[a4paper,11pt]{article}
\pagenumbering{arabic}
\usepackage{../environment}
\usepackage{listings}
\usepackage{color}
\usepackage{fancyvrb} % to have verbatim input from solutions.py


\definecolor{dkgreen}{rgb}{0,0.6,0}
\definecolor{gray}{rgb}{0.5,0.5,0.5}
\definecolor{mauve}{rgb}{0.58,0,0.82}

\lstset{frame=tb,
  language=python,
  aboveskip=3mm,
  belowskip=3mm,
  showstringspaces=false,
  columns=flexible,
  basicstyle={\small\ttfamily},
  numbers=none,
  numberstyle=\tiny\color{gray},
  keywordstyle=\color{blue},
  commentstyle=\color{dkgreen},
  stringstyle=\color{mauve},
  breaklines=true,
  breakatwhitespace=true,
  tabsize=3
}

\begin{document}

\begin{center}
    \huge{Solutions to Sheet 6}
\end{center}

\exercise{1}
Let $K$ be a number field. Show that $\cO_K$ has infinitely many
prime ideals.

\textbf{Solution.}
There are many ideas one could use. For example, the statement is a direct 
consequence of the lying over theorem for integral extensions. But we proof this
mimicking euclid's proof. Assume there is only a finite number of primes
$\fp_1, \dots, \fp_n$. Let $n \in \Z$ be an integer such that 
$n\Z = \fp_1 \cdots \fp_n \cap \Z$. Now $(n+1)\cO_K$ is a prime ideal not contained
in any of the ideals $\fp_1, \dots, \fp_n$. In particular, decomposition into prime ideals
cannot hold. This is a contradiction.

\exercise{2}
Let $m \in \Z$ be negative and squarefree with $m \equiv 1$ mod $4$ and
set $K = \Q(\sqrt m)$. We assume that $\cO_K$ is a UFD (this is 
used in parts (ii) and (iv)).
\begin{enumerate}
\item Let $p$ be a prime number and $k \in \Z$ such that 
    $p \mid k^2 - k + \frac{1-m}4$. Show that $p$ is not a 
    prime element in $\cO_K$.
\item Let $p$ be as in $(i)$. Show that there exists $u,v$ in $\cO_K$
    such that $p \equiv uv$ and $\NormKQ(u) = p$. 
\item Let $p$ be a prime number of the form $\NormKQ(u)$ 
    for some $u \in \cO_K$. Show that $p \geq (1-m)/4$. 
\item Suppose that $m < -3$. Deduce that every number of the form
    $k^2 - k + \frac{1-m}4$ with $0 \leq k \leq \frac{-3-m}4$ is prime.
\end{enumerate}
\textbf{Solution.}
\begin{enumerate}[labelindent=0pt, wide]
    \item Let $\alpha = \left(\frac{1+\sqrt m}2\right)$. Then we can factor
        $k^2 - k + \frac{1-m}4 = (k - \alpha)(k - \sigma(\alpha))$, where $\sigma$ is
        complex conjugation (in particular, 
        $k^2-k+\frac{1-m}4 = \NormKQ(k-\alpha)$). We know that $(1,\alpha)$ is
        a $\Z$-basis for $\cO_K$, and wee see that $k-\alpha, k-\sigma(\alpha)
        \not \in p\cO_K$. Hence $p\cO_K$ is not a prime ideal, and $p$ is
        not prime.
    \item We make use of the fact that $\cO_K$ is a UFD. Let $p = q_1 \dots q_r$ be 
        a decomposition of $p$ into (possibly repeating) irreducible factors
        (without units).
        Then $p^2 = \NormKQ(p) = \NormKQ(q_1) \cdots \allowbreak \NormKQ(q_r)$,
        and we find that
        $r \leq 2$. As $\cO_K$ is a UFD, $p$ is not irreducible (prime =
        irreducible in UFDs). This shows that $r \geq 2$, so we
        have equality, and we get two elements $q_1, q_2$ with $\NormKQ(q_1) = 
        \NormKQ(q_2) = p$. 
    \item Write $u = a + b \alpha$. Then 
        \begin{equation*}
            \NormKQ(u) = \left( a + \frac b2\right)^2 - \frac{b^2}4m \geq \frac{1-m}4.
        \end{equation*}
        Here we used that necessarily $b\neq0$ if this is suppsed to be prime. Also,
        note that both terms are non-negative.
    \item Suppose $p_1$ and $p_2$ are prime numbers that divide $k^2 - k + \frac{1-m}4$. 
        By 2. there are $u_1,u_2$ in $\cO_K$ such that $\NormKQ(u_i) = p_i$. In particular
        we find by 3. that $p \geq \frac{1-m}4$. Now as $m < -3$, we find that 
        $$p_1 p_2 \geq \left (\frac{1-m}4 \right)^2 \leq k^2 - k + \frac{1-m}4.$$
        The last inequality rewrites as
        \begin{equation*}
            \left(\frac{1-m}4\right) \left(\frac{1-m}4 - 1 \right) \leq
                k(k-1),
        \end{equation*}
        which is only possible if $k \geq \frac{1-m}4$ or $k < 0$.
\end{enumerate}
\textbf{Remark.} The last statement implies the funny result that $k^2 - k + 41$
is a prime for all integers $0 \leq k < 41$, as $\cO_{\Q(\sqrt{-163})}$ is known 
to be a UFD.


\exercise{3}
Let $K$ be a number field. Let $I$ and $J$ be ideals of 
$\cO_K$ and let $\sigma: K \to K$ be a field automorphism. Recall that 
$\sigma(\cO_K) \subset \cO_K$.
\begin{enumerate}
    \item Show that $\sigma(I)$ is an ideal of $\cO_K$.
    \item Show that $\sigma(I)$ is prime if $I$ is prime.
    \item Show that $\sigma(IJ) = \sigma(I)\sigma(J)$.
\end{enumerate}
\textbf{Solution.}
\begin{enumerate}[labelindent=0pt, wide]
    \item For $x \in I$, $r \in \cO_K$ we have 
        $$r\sigma(x) = \sigma (\sigma^{-1}(r) x) \in \sigma(I).$$
        Hence $\sigma(I)$ is an ideal.
    \item Same trick: If $I$ is prime and $xy \in \sigma(I)$, then 
        $\sigma^{-1}(x) \sigma^{-1}(y) \in I$, so by primality of $I$ 
        and without loss of generality $\sigma^{-1}(x) \in I.$ But now 
        $x \in \sigma(I)$, so $\sigma(I)$ is prime. 
    \item $\sigma(IJ) = \{\sigma(x)\sigma(y) \mid x \in I, y \in J\} =
        \sigma(I) \sigma(J).$
\end{enumerate}


\exercise{4}
Let $R$ be a Dedekind domain.
\begin{enumerate}
    \item Let $I$ and $I_1, \dots, I_n$ be ideals such that 
        $I_j \nmid I$ for all $j = 1, \dots, n$. Show that 
        \begin{equation*}
            I\setminus(I_1 \cup \dots \cup I_n) \neq \emptyset.
        \end{equation*}
    \item Suppose that $R$ has at most finitely many prime ideals. 
        Show that $R$ is a principal ideal domain.
\end{enumerate}
\textbf{Solution.}
The following lemma will prove useful (and is really just a weak form of 
4.1):
\begin{lem}\label{lem:1}
    Let $R$ be a Dedekind domain and let $\fp_1, \dots, \fp_n$ prime ideals of $R$.
    Let $e_1 \dots, e_n \in \Z$ be arbitrary integers. Then there is some 
    $r \in R$ with $r \in \fp_j^{e_j}\setminus \fp_j^{e_{j+1}}$ for all $j$. 
\begin{proof}
    We'll make use of the Chinese remainder theorem. We have the map
    \begin{equation*}
        R \to R/(\fp_1^{e_1+1} \cap \dots \fp_n^{e_n+1}) \cong
        \prod_j R/\fp_j^{e_j+1}.
    \end{equation*}
    Now choose non-zero elements $s_j \in \fp_j^{e_j}/\fp_j^{e_j+1} \subset
    R/\fp_j^{e_j+1}$. Any element $r$ in the preimage of 
    \begin{equation*}
        (s_1, \dots, s_n) \in \prod_j R/\fp_j^{e_j+1}
    \end{equation*}
    works.
\end{proof}
\end{lem}
\begin{enumerate}
    \item We are in the Dedekind situation, so of course we look at the prime
        factorization of the Ideals at hand. Let 
        $I = \fp_1^{e_1} \dots \fp_m^{e_m}$. Also, by the divisibility
        assumption, for any $j$ there is some prime ideal $\fq_j$ and some integer $f_j$ 
        such that $\fq_j^{f_j} \mid I$, $\fq_j^{f_j+1} \nmid I$ and 
        $\fq_j^{f_j+1} \mid I_j$. Now, there is some element 
        $r \in R$ with $r \in \fp_i^{e_i}$ for all $i$ (i.e., $r \in I$) and 
        $r \in \fq_j^{e_j}\setminus\fq_j^{e_j+1}$
        (i.e., $r \not \in I_j$).

    \item As $R$ is a Dedekind domain, it suffices to show that all
        prime ideals are principal. By assumption there are only finitely many,
        let's call them $\fp_1, \dots, \fp_n$. We now use lemma $1$ to find an
        element $x \in R$ with $x \not \in \fp_j$ for $j \neq 1$ and 
        $x \in \fp_1 \setminus\fp_1^2$. This forces $(x) = \fp_1$. 
\end{enumerate}

\contactend
\end{document}
